%!TEX root = HibersapReference.tex

\chapter{Configuration}
\label{cha:Configuration}

Hibersap configuration consists of a set of properties. 
There are two possibilities to configure Hibersap: 
\begin{itemize}
  \item XML file configuration 
  \item Programmatic configuration
\end{itemize}

While bootstrapping, Hibersap first tries to configure itself looking for the /META-INF/hibersap.xml file in the
classpath. When creating a SessionManager, configuration can be set or overwritten using programmatic configuration.

Using an XML file is the preferred way to configure Hibersap.


\section{XML file configuration}
\label{sec:XML file configuration}

The format of the XML configuration file is heavily inspired by JPA's persistence.xml.
If you have to access different SAP systems, you can define more than one SessionManager by multiplying the
session-manager XML element.

\begin{lstlisting}[language=XML,caption=hibersap.xml]
<?xml version="1.0" encoding="UTF-8"?>
<hibersap>
  <session-manager name="A12">
    <context>org.hibersap.execution.jco.JCAContext</context>

    <jca-connection-factory>java:/eis/sap/A12</jca-connection-factory>
    <jca-connectionspec-factory>
      org.hibersap.execution.jca.cci.SapBapiJcaAdapterConnectionSpecFactory
    </jca-connectionspec-factory>

    <properties>
      <property name="jco.client.client" value="800" />
      <property name="jco.client.user" value="sapuser" />
      <property name="jco.client.passwd" value="password" />
      <property name="jco.client.lang" value="en" />
      <property name="jco.client.ashost" value="10.20.80.76" />
      <property name="jco.client.sysnr" value="00" />
    </properties>

    <annotated-classes>
      <class>org.hibersap.examples.flightlist.FlightListBapi</class>
      <class>org.hibersap.examples.flightdetail.FlightDetailBapi</class>
    </annotated-classes>
    
    <interceptor-classes>
      <class>com.mypackage.MyHibersapInterceptor</class>
    <interceptor-classes>
  </session-manager>
</hibersap>
\end{lstlisting}

Table \ref{tab:ConfigurationParameters} shows a list of the configuration parameters.

\begin{table}[H]
  \renewcommand{\arraystretch}{1.5}
  \centering
   \begin{tabularx}{\textwidth}{ l X }
    \toprule
    \textbf{Parameter}      & \textbf{Description} \\ 
    \midrule
    context                 &     
    The fully qualified class name of the Context class. This class must implement org.hibersap.session.Context and acts as a facade to
    the actually used interfacing technology. Existing 
    implementations are org.hibersap.execution.jco.JCoContext for the SAP Java Connector (JCo) or org.hibersap.execution.jca.JCAContext 
    for a JCA compatible Resource Adapter. (Default: JCoContext.) 
    \\
    jca-connection-factory  & 
    The JNDI name of the JCA Connection Factory. This parameter has to be specified if the application uses a JCA compatible resource 
    adapter. The resource adapter has to be deployed on the application server independently from Hibersap and the application, defining 
    a JNDI name. Hibersap will use this name to look up the resource adapter's ConnectionFactory. 
    \\
    jca-connectionspec-factory &
    The fully qualified class name of the ConnectionSpecFactory implementation used to get a ConnectionSpec object. A ConnectionSpec is
    is used to provide data such as a user name and password for the current session. An existing implementation is 
    SapBapiJcaAdapterConnectionSpecFactory.
    \\
    properties                   & 
    Zero or more additional properties. These depend on the interfacing technology in use. For the SAP JCo, all the JCo-specific 
    properties must be defined here. 
    \\
    annotated-classes        & 
    All annotated BAPI classes which are used with the SessionManager must be listed here. 
    \\
    interceptor-classes       & 
    A list of the interceptor classes. These must implement org.hibersap.session.ExecutionInterceptor and are called before and after a 
    SAP function module gets executed. \\ 
    \bottomrule
  \end{tabularx}
  \caption{Hibersap configuration parameters}
  \label{tab:ConfigurationParameters}
\end{table}

To build a SessionManager using the hibersap.xml file, you simply have to create an object of
class org.\-hibersap.\-configuration.\-AnnotationConfiguration, specifying the SessionManager name as an argument. Note 
that there is also a default constructor for AnnotationConfiguration which can be used if there 
is only one configured SessionManager. Hibersap will issue a warning when there are more than one SessionManagers 
configured, but the no-args constructor is used. 

\begin{lstlisting}[label=hibersap.xml]
AnnotationConfiguration configuration= new AnnotationConfiguration("A12");
SessionManager sessionManager = configuration.buildSessionManager();
\end{lstlisting}


\section{Programmatic configuration}
\label{sec:Programmatic configuration}

After creating a Configuration object which will be used to build the SessionManager, configuration can be set or
overwritten programmatically.

The information from the XML file is internally converted into a Java data structure reflecting the structure of the
XML document. All configuration classes have Java Bean style accessor methods (get.. and set..) for
their fields. 

For programmatic configuration you have to change or create SessionManagerConfig for each 
SessionManager you want to use. You can use method chaining to build the
object. The following example creates a  SessionManagerConfig object that is equal to the one created internally by the
hibersap.xml example in section \ref{sec:XML file configuration}. 

\begin{lstlisting}[caption=Programmatic configuration]
SessionManagerConfig cfg = new SessionManagerConfig( "A12" )
    .setContext( JCoContext.class.getName() )
    .setJcaConnectionFactory( "java:/eis/sap/A12" )
    .setJcaConnectionSpecFactory( "org.hibersap.execution.jca.cci." 
                              + "SapBapiJcaAdapterConnectionSpecFactory" )
    .setProperty( "jco.client.client", "800" )
    .setProperty( "jco.client.user", "sapuser" )
    .setProperty( "jco.client.passwd", "password" )
    .setProperty( "jco.client.lang", "en" )
    .setProperty( "jco.client.ashost", "10.20.80.76" )
    .setProperty( "jco.client.sysnr", "00" )
    .addAnnotatedClass( FlightListBapi.class )
    .addAnnotatedClass( FlightDetailBapi.class )
    .addInterceptor( MyHibersapInterceptor.class );

AnnotationConfiguration configuration = new AnnotationConfiguration(cfg);
SessionManager sessionManager = configuration.buildSessionManager();
\end{lstlisting}








