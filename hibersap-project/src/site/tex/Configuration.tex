%!TEX root = userguide.tex
\chapter{Configuration}
\label{cha:Configuration}

Hibersap configuration consists of a set of properties. 
There are three possibilities to configure Hibersap: 
\begin{itemize}
  \item XML file configuration 
  \item Properties file configuration 
  \item Programmatic configuration
\end{itemize}

While bootstrapping, Hibersap first tries to configure itself looking for the /META-INF/hibersap.xml file in the
classpath. If (and only if) the XML file was not found or does not contain any properties, Hibersap tries to locate the
hibersap.properties file in the classpath and read the properties from there. When creating a SessionFactory,
properties can be set or overwritten using programmatic configuration.


Using an XML file is the preferred way to configure Hibersap in an application.





\section{XML file configuration}

The format of the XML configuration file is heavily inspired by JPA's persistence.xml.
If you have to access different SAP systems, you can define more than one SessionFactory by multiplying the
session-factory XML element.

\begin{Verbatim}[frame=single,label=hibersap.xml]
<?xml version="1.0" encoding="UTF-8"?>
<hibersap>
  <session-factory name="A12">
    <context>org.hibersap.execution.jco.JCoContext</context>

    <properties>
      <property name="jco.client.client" value="800" />
      <property name="jco.client.user" value="sapuser" />
      <property name="jco.client.passwd" value="password" />
      <property name="jco.client.lang" value="en" />
      <property name="jco.client.ashost" value="10.20.80.76" />
      <property name="jco.client.sysnr" value="00" />
      <property name="jco.destination.pool_capacity" value="1" />
    </properties>

    <class>org.hibersap.examples.flightlist.FlightListBapi</class>
    <class>org.hibersap.examples.flightdetail.FlightDetailBapi</class>
  </session-factory>
</hibersap>
\end{Verbatim}

\section{Properties file configuration}

Note that using a properties file it is not possible to define more than one SessionFactory. If you have to access more
than one SAP system in your application, you have to provide an XML properties file or manual configuration.

\begin{Verbatim}[frame=single,label=hibersap.properties]
# The session factory name
hibersap.session_factory_name=A12

# The context class
hibersap.context_class=org.hibersap.execution.jco.JCoContext

# Java Connector settings
jco.client.client=800
jco.client.user=sapuser
jco.client.passwd=password
jco.client.lang=en
jco.client.ashost=10.20.80.76
jco.client.sysnr=00
jco.destination.pool_capacity=1

# Mapped classes
hibersap.bapi_class.0=org.hibersap.examples.flightlist.FlightListBapi
hibersap.bapi_class.1=org.hibersap.examples.flightdetail.FlightDetailBapi
\end{Verbatim}


\section{Programmatic configuration}

After creating a Configuration object
which is used to build the SessionFactory, properties can be set or overwritten using the methods
Configuration.setProperty(String, String) or Configuration.setProperties(Properties).

Using the setProperty() method of Hibersap's AnnotationConfiguration class, you can specify the properties
programmatically. This method can be used to set properties that are not specified in a hibersap configuration file, but
can also be used to overwrite properties defines there. This can be very convenient for a test setup, because you
don't have to care for different properties files to be in place for different runtime environments, like test,
integration and production.




