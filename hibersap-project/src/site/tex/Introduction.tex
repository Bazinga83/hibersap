%!TEX root = userguide.tex
\chapter{Introduction to Hibersap}

  Hibersap is a small framework that offers an abstraction layer on top of the SAP Java Connector (JCo).
  It maps Java classes to SAP function modules using Java Annotations and reduces the technical code
  to call a function in a SAP back-end system to a minimum. 
  Hibersap's API is very similar to Hibernate, thus offering a familiar programming interface to developers.  

\section*{Current Features}

\begin{itemize}
  \item Calling remote function modules in SAP systems
  \item Mapping of ABAP function modules and their parameters to Java classes and their fields using Java Annotations.
  \item Hibernate-like API completely hiding the JCo/JCA API.
  \item Custom data type conversion using @Convert annotation and Converter classes.
  \item Intercept execution of function modules using custom Interceptors.
  \item Automatic error propagation using the @ThrowExceptionOnError annotation.
  \item Configuration using XML file or programmatic configuration
  \item Using the SAP Java Connector (JCo), version 3, internally.
  \item Alteratively, using a JCA compliant Resource Adapter in managed environments.
  \item Switching between the use of JCo and JCA by means of configuration.
  \item Manual transaction handling using JCo.
  \item Local Transactions using JCA.
  \item Container Managed Transactions (CMT) with Application Servers that support Last Resource Commit Optimization.
\end{itemize}


\section*{Prospective Features}

\begin{itemize}
  \item Automated lookup of mapped classes instead of adding them manually
  \item JBoss5 deployer for annotated Hibersap classes.
  \item Hibernate-like current session context strategy.
  \item Integration of the Hibernate Validation Framework to validate mapped fields.  
  \item Using JCo 3 stateless calls.
\end{itemize}

In chapter \ref{cha:QuickStart} you'll find out how easy it is to use Hibersap.
